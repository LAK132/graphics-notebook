\documentclass{article}
\usepackage{listings} % For \lstset and \lstinputlisting.
\usepackage{xcolor} % For \pagecolor and \color.
\usepackage{hyperref} % For \url and \href

\definecolor{pale-red}{HTML}{FF5555}
\definecolor{dark-red}{HTML}{802525}

\newcommand{\example}[3][\footnotesize]{
\lstset{basicstyle=#1\ttfamily,
        language=#2,
        keywordstyle=\color{pale-red},
        stringstyle=\bfseries\color{dark-red},
        identifierstyle=\color{blue}}
\lstinputlisting{examples/#3}}

\newcommand{\textem}[1]{\textit{\textbf{#1}}}

\title{LAK132's Graphics Notebook}
\author{LAK132}

\begin{document}

\maketitle

This is a living document.

My choice of the word ``Notebook'' is very intentional,
I am using this document as a singluar place to document my thoughts about
writing graphical software in the modern age.

I have chosen to use this form over a regular paper notebook as it will allow
me to restructure my thoughts as time goes on.

I do not make any claim that anything here is factual,
please take everything with a grain of salt.

With that said, I am publishing this notebook in the hope that it may help
others navigate the \textem{utter hellscape} that is modern graphical
software design.

\pagebreak
\tableofcontents

% -----------------------------------------------------------------------------
% Preamble
% -----------------------------------------------------------------------------

\pagebreak
\section{Preable}
\label{sec:preamble}

\subsection{Dependencies}
\label{subsect:dependencies}

Dependencies on modern computers are hell.
There's no simpler way to put it.
Sure, depending on some kinds of libraries is a neccesety given the
unfathomable number of hardware combinations used in modern computers.
But we shouldn't be using this as an excuse librarify and then dynamic link to
every single thing possible.
\\
My goal here is to explore the process of building graphical programs for
modern computers with the fewest number of dependencies.
The good news is that as far as I can tell,
\textem{it is entirely possible} to build simple modern software that
depends on nothing but the OS it is running under, a windowing library and
optionally a graphics acceleration library

\subsection{Abstractions}
\label{subsec:abstractions}

Any attempt to reduce the number of dependencies we need will likely results in
a reduction in the number of abstractions we have to deal with.
Academia would likely tell you this is a bad idea,
that abstractions are required so that mere humans could have a fighting chance
of understanding the code they write.
\\
This is false.
I have dealt with code where reducing from 4 classes,
each abstracting a particular behaviour,
to 1 class with a couple of conditional parameters has \textem{increased}
understandability of the code over all.
\\
Abstractions have their place though.
If we want our code to be ``portable'',
then it is neccesary that we abstract away the \textit{commonanlities} of the
systems we wish to target.
For example, Xlib (\ref{subsubsec:xlib}), XCB (\ref{subsubsec:xcb}) and Win32
(\ref{subsec:win32}) all use a similar process for obtaining a connection to
the display, and getting a graphics context for the display.
We could easily create a ``generic'' interface for this that our application
code targets, then have platform dependant code that translates this (with
minimal overhead) into the platform specific interface.

% -----------------------------------------------------------------------------
% Windowing
% -----------------------------------------------------------------------------

\pagebreak
\section{Windowing}
\label{sec:windowing}

% -----------------------------------------------------------------------------
% X11
% -----------------------------------------------------------------------------

\subsection{X11}
\label{subsec:x11}

X11 is the 11th iteration of the X windowing system.
The X windowing system was developed for Unix workstations in the 1980s,
and continues to be used on modern *nix machines today.

\subsubsection{Xlib}
\label{subsubsec:xlib}

Xlib is one of the most straight forward libraries for managing windows
(sec~\ref{subsec:simple/xlib}).

\subsubsection{XCB}
\label{subsubsec:xcb}

The primary difference between Xlib and XCB is that Xlib must do a round trip
to the X server after every command,
whereas XCB buffers commands asynchronously (sec~\ref{subsec:simple/xcb}).

% -----------------------------------------------------------------------------
% Wayland
% -----------------------------------------------------------------------------

\subsection{Wayland}
\label{subsec:wayland}

Wayland is a windowing system developed by freedesktop.org which is intended to
replace the ancient X windowing system.
Wayland isn't widely supported yet.
(sec~\ref{subsec:simple/wayland})

% -----------------------------------------------------------------------------
% Win32
% -----------------------------------------------------------------------------

\subsection{Win32}
\label{subsec:win32}

Win32 is the API programs use to talk to recent Windows operating systems.
Programs must use the Win32 API to open graphical windows
(sec~\ref{subsec:simple/win32}).

% -----------------------------------------------------------------------------
% Rendering
% -----------------------------------------------------------------------------

\pagebreak
\section{Rendering}
\label{sec:rendering}

% -----------------------------------------------------------------------------
% OpenGL
% -----------------------------------------------------------------------------

\subsection{OpenGL}
\label{subsec:opengl}

% -----------------------------------------------------------------------------
% Vulkan
% -----------------------------------------------------------------------------

\subsection{Vulkan}
\label{subsec:vulkan}

% -----------------------------------------------------------------------------
% DirectX
% -----------------------------------------------------------------------------

\subsection{DirectX}
\label{subsec:directx}

% -----------------------------------------------------------------------------
% Direct3D
% -----------------------------------------------------------------------------

\subsection{Direct3D}
\label{subsec:direct3d}

% -----------------------------------------------------------------------------
% Direct2D
% -----------------------------------------------------------------------------

\subsection{Direct2D}
\label{subsec:direct2d}

% -----------------------------------------------------------------------------
% GDI
% -----------------------------------------------------------------------------

\subsection{GDI}
\label{subsec:windows-gdi}

% -----------------------------------------------------------------------------
% Software rasterisation
% -----------------------------------------------------------------------------

\subsection{Software rasterisation}
\label{subsec:software-rasterisation}

% -----------------------------------------------------------------------------
% Xlib blitting
% -----------------------------------------------------------------------------

\subsubsection{Xlib blitting}
\label{subsubsection:xlib-blitting}

(sec~\ref{subsec:blit/xlib})

% -----------------------------------------------------------------------------
% XCB blitting
% -----------------------------------------------------------------------------

\subsubsection{XCB blitting}
\label{subsubsection:xcb-blitting}

(sec~\ref{subsec:blit/direct_xcb})
\\
(sec~\ref{subsec:blit/pixmap_xcb})

% -----------------------------------------------------------------------------
% Examples
% -----------------------------------------------------------------------------

\pagebreak
\section{Examples}
\label{sec:examples}

% -----------------------------------------------------------------------------
% Build scripts for GNU toolchain
% -----------------------------------------------------------------------------

\subsection{Build scripts for GNU toolchain}
\label{subsec:build-scripts-gnu}

\subsubsection{Makefile}
\label{subsubsec:makefile}

\example{Make}{Makefile}

% -----------------------------------------------------------------------------
% Build scripts for MSVC toolchain
% -----------------------------------------------------------------------------

\subsection{Build scripts for MSVC toolchain}
\label{subsec:build-scripts-msvc}

\subsubsection{make.bat}
\label{subsubsect:make.bat}

This script is intended to minimally replicate GNU-Make on Windows.

\example[\tiny]{command.com}{make.bat}

\subsubsection{makelist.bat}
\label{subsubsect:makelist.bat}

This script is intented to minimally replicate a GNU-Make Makefile on Windows.

\example[\tiny]{command.com}{makelist.bat}

% -----------------------------------------------------------------------------
% simple/xlib
% -----------------------------------------------------------------------------

\subsection{simple/xlib}
\label{subsec:simple/xlib}

X11 example using Xlib to open a window and draw some rectangles and a string
in it.

\example{C}{simple/xlib.c}

% -----------------------------------------------------------------------------
% blit/xlib
% -----------------------------------------------------------------------------

\subsection{blit/xlib}
\label{subsec:blit/xlib}

X11 example using Xlib to open a window and blit an image into it at the
cursor. \verb|-DSTATIC_IMAGE_BLIT| for statically allocated pixel buffer.

\example{C}{blit/xlib.c}

% -----------------------------------------------------------------------------
% simple/xcb
% -----------------------------------------------------------------------------

\subsection{simple/xcb}
\label{subsec:simple/xcb}

X11 example using XCB to open a window and draw some rectangles and a string
in it.

\example{C}{simple/xcb.c}

% -----------------------------------------------------------------------------
% blit/direct_xcb
% -----------------------------------------------------------------------------

\subsection{blit/direct\_xcb}
\label{subsec:blit/direct_xcb}

X11 example using XCB to open a window and blit an image into it at the
cursor.
This example directly blits from the pixel buffer to the window.

\example{C}{blit/direct_xcb.c}

% -----------------------------------------------------------------------------
% blit/pixmap_xcb
% -----------------------------------------------------------------------------

\subsection{blit/pixmap\_xcb}
\label{subsec:blit/pixmap_xcb}

X11 example using XCB to open a window and blit an image into it at the
cursor.
This example blits the pixel buffer to a pixmap,
then blits the pixmap to the window.
\verb|-DUSE_XCB_IMAGE| to use \verb|xcb/xcb_image.h|.

\example{C}{blit/pixmap_xcb.c}

% -----------------------------------------------------------------------------
% simple/wayland
% -----------------------------------------------------------------------------

\subsection{simple/wayland}
\label{subsec:simple/wayland}

Untested Wayland example (I don't have a computer that runs Wayland).
Based on \url{https://github.com/eyelash/tutorials/blob/master/wayland-egl.c}.

\example{C}{simple/wayland.c}

% -----------------------------------------------------------------------------
% simple/win32
% -----------------------------------------------------------------------------

\subsection{simple/win32}
\label{subsec:simple/win32}

Compiling on Linux
\url{https://eggplant.pro/blog/writing-win32-applications-with-mingw-and-wine/}
\\
Win32 example to open a window and draw some rectangles and a string in it.

\example{C++}{simple/win32.cpp}

% -----------------------------------------------------------------------------
% blit/win32
% -----------------------------------------------------------------------------

\subsection{blit/win32}
\label{subsec:blit/win32}

Compiling on Linux
\url{https://eggplant.pro/blog/writing-win32-applications-with-mingw-and-wine/}

\example{C++}{blit/win32.cpp}

\end{document}
