\documentclass{article}
\usepackage{verbatim} % Limit \verbatiminput files to 65 columns.
\usepackage{xcolor} % For \pagecolor and \color.
\usepackage{hyperref} % For \url and \href

\title{LAK132's Graphics Notebook}
\author{LAK132}

\begin{document}

\maketitle

This is a living document.

My choice of the word "Notebook" is very intentional,
I am using this document as a singluar place to document my thoughts about
writing graphical software in the modern age.

I have chosen to use this form over a regular paper notebook as it will allow
me to restructure my thoughts as time goes on.

I do not make any claim that anything here is factual,
please take everything with a grain of salt.

With that said, I am publishing this notebook in the hope that it may help
others navigate the \textit{\textbf{utter hellscape}} that is modern graphical
software design.

\tableofcontents

% -----------------------------------------------------------------------------
% Preamble
% -----------------------------------------------------------------------------

\section{Preable}
\label{sec:preamble}

Dependencies on modern computers are hell.
There's no simpler way to put it.
Sure, depending on some kinds of libraries is a neccesety given the
unfathomable number of hardware combinations used in modern computers.
But we shouldn't be using this as an exuse librarify and then dynamic link to
every single thing possible.
\\
My goal here is to explore the process of building graphical programs for
modern computers with the fewest number of dependencies.
The good news is that as far as I can tell,
\textit{\textbf{it is entirely possible}} to build simple modern software that
depends on nothing but the OS it is running under, a windowing library and
optionally a graphics acceleration library

% -----------------------------------------------------------------------------
% Windowing
% -----------------------------------------------------------------------------

\section{Windowing}
\label{sec:windowing}

% -----------------------------------------------------------------------------
% X11
% -----------------------------------------------------------------------------

\subsection{X11}
\label{subsec:x11}

X11 is the 11th iteration of the X windowing system.
The X windowing system was developed for Unix workstations in the 1980s,
and continues to be used on modern *nix machines today.

\subsubsection{Xlib}
\label{subsubsec:xlib}

Xlib is one of the most straight forward libraries for managing windows
(sec~\ref{subsec:simple_xlib.c}).

\subsubsection{XCB}
\label{subsubsec:xcb}

The primary difference between Xlib and XCB is that Xlib must do a round trip
to the X server after every command,
whereas XCB buffers commands asynchronously (sec~\ref{subsec:simple_xcb.c}).

% -----------------------------------------------------------------------------
% Wayland
% -----------------------------------------------------------------------------

\subsection{Wayland}
\label{subsec:wayland}

Wayland is a windowing system developed by freedesktop.org which is intended to
replace the ancient X windowing system.
Wayland isn't widely supported yet.
(sec~\ref{subsec:simple_wayland.c})

% -----------------------------------------------------------------------------
% Win32
% -----------------------------------------------------------------------------

\subsection{Win32}
\label{subsec:win32}

Win32 is the API programs use to talk to recent Windows operating systems.
Programs must use the Win32 API to open graphical windows
(sec~\ref{subsec:simple_win32.cpp}).

% -----------------------------------------------------------------------------
% Rendering
% -----------------------------------------------------------------------------

\section{Rendering}
\label{sec:rendering}

% -----------------------------------------------------------------------------
% OpenGL
% -----------------------------------------------------------------------------

\subsection{OpenGL}
\label{subsec:opengl}

% -----------------------------------------------------------------------------
% Vulkan
% -----------------------------------------------------------------------------

\subsection{Vulkan}
\label{subsec:vulkan}

% -----------------------------------------------------------------------------
% DirectX
% -----------------------------------------------------------------------------

\subsection{DirectX}
\label{subsec:directx}

% -----------------------------------------------------------------------------
% Direct3D
% -----------------------------------------------------------------------------

\subsection{Direct3D}
\label{subsec:direct3d}

% -----------------------------------------------------------------------------
% Direct2D
% -----------------------------------------------------------------------------

\subsection{Direct2D}
\label{subsec:direct2d}

% -----------------------------------------------------------------------------
% GDI
% -----------------------------------------------------------------------------

\subsection{GDI}
\label{subsec:windows-gdi}

% -----------------------------------------------------------------------------
% Software rasterisation
% -----------------------------------------------------------------------------

\subsection{Software rasterisation}
\label{subsec:software-rasterisation}

% -----------------------------------------------------------------------------
% Xlib blitting
% -----------------------------------------------------------------------------

\subsubsection{Xlib blitting}
\label{subsubsection:xlib-blitting}

% -----------------------------------------------------------------------------
% Examples
% -----------------------------------------------------------------------------

\section{Examples}
\label{sec:examples}

\subsection{simple\_xlib.c}
\label{subsec:simple_xlib.c}

X11 example using Xlib to open a window and draw some rectangles and a string
in it:

\verbatiminput{examples/simple_xlib.c}

\subsection{simple\_xcb.c}
\label{subsec:simple_xcb.c}

X11 example using XCB to open a window and draw some rectangles and a string
in it:

\verbatiminput{examples/simple_xcb.c}

\subsection{simple\_wayland.c}
\label{subsec:simple_wayland.c}

Untested Wayland example (I don't have a computer that runs Wayland).
Based on \url{https://github.com/eyelash/tutorials/blob/master/wayland-egl.c}.

\verbatiminput{examples/simple_wayland.c}

\subsection{simple\_win32.cpp}
\label{subsec:simple_win32.cpp}

Win32 example to open a window and draw some rectangles and a string in it:

\verbatiminput{examples/simple_win32.cpp}

\end{document}
